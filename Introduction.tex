The recently discovered Higgs boson at the LHC could be the final particle required in the highly successful theory of the Standard Model. However, the SM is not without its shortcomings, for instance one requires fine-tuned cancellations of large quantum corrections for the Higgs boson to have a mass at the electroweak symmetry breaking scale known as the hierarchy problem. The magnitude of this fine-tuning makes one suspect that there is some dynamical mechanism which makes this fine-tuning ``natural''. SUSY, a popular extension of the SM offers a well-motivated explanation: the dominant role in cancelling the quantum effects would come from scalar partners to the bottom/top quarks (sbottoms/stops), due to the large 3rd generation Yukawa couplings in the SM, and fermionic partners to the Higgs boson (higgsinos). The lightest SUSY particle, the neutralino, could also be a candidate for Dark Matter. Alternatively, the naturalness of electroweak symmetry breaking could be effectively explained by non-SUSY BSM models involving strong dynamics. Searching for sparticles in the all-hadronic mode is very attractive because it makes up a large portion of branching fractions for typical signals. The final state involves events with high jet multiplicity and missing transverse energy and this search method could eventually decipher the mysteries surrounding the Higgs boson.\\

The work presented in this thesis is about discovering physics beyond the Standard Model and is three fold: (1) A simulation study is performed to determine the effect of CMS Phase-2 Pixel Detector on stubs (matching hit pairs from the same particle track on two adjacent layers of silicon modules) (2) An analysis involving a search for Supersymmetry in the all-hadronic channel with missing transverse momentum and a customized top tagger (3) An improved new method based on $\gamma$+jets to determine Z$\rightarrow\nu \bar{\nu}$ background in search for the SUSY signature.\\

This thesis work was carried out at the LPC (LHC Physics Center) at Fermilab. The funding for this work was provided by the \textbf{NSF (National Science Foundation - NSF Award ID 1506168) and LPC (Guest and Visitor) Fellowship}.\\

\Cref{LHCCMS} presents a basic description of the Large Hadron Collider and CMS Detector (including trigger, data acquisition and Phase-2 upgrade plans), which are the basic tools used to collect data in order to successfully complete the thesis work.\\ 

\Cref{theory} gives a brief motivation to look for extensions for physics Beyond the Standard Model for which Supersymmetry (SUSY) is the most popular choice. It describes SUSY briefly and in relation to Simplified Models few of which are used in the analysis described in the thesis.\\

\Cref{stubs} is dedicated to a study optimizing the geometry of Phase-2 Pixel Detector for the High Luminosity-LHC.  A simulation is performed using the t$\bar{\text{t}}$ sample to estimate the impacting of pixel detector on the formation of stubs used in the outer tracker to determine particle tracks that are used in the Level-1 (L1) trigger. This work was done as part of service work for the CMS experiment.\\

\Cref{AnalysisChap} presents an analysis on the search for SUSY in the 0-Lepton final state where an inclusive search for events with final states that contain missing transverse momentum and reconstructed top quarks is performed. The signal models used in this study include the production of three different types of SUSY particles. Two of which are the top squark and the gluino, the supersymmetric partners of the SM top and gluon, respectively. The third one is the neutralino that is considered to be the lightest SUSY particle (LSP) under the Minimal Supersymmetric Standard Model (MSSM) and a possible candidate for Dark Matter. Background from t$\bar{\text{t}}$, single top quark, W+jets and Z$\rightarrow\nu \bar{\nu}$ events was estimated.\\

\Cref{estimation} details an improved new method based on $\gamma$+jets to determine Z$\rightarrow\nu \bar{\nu}$ background in search for the SUSY signature. This estimation procedure builds upon the Z$\rightarrow\nu \bar{\nu}$ method used in SUSY analysis described in Chapter 5 (also referred to as 2016 analysis) and aims to refine the overall background calculation as well as reduce the uncertainties associated with the previous method. To accomplish this, an additional $\gamma$+jets control sample (CS) is used in conjunction with the tight Z$\rightarrow\mu^{+}\mu^{-}$ control region used in the 2016 analysis estimation. The new  $\gamma$+jets CS provides a more data-driven estimation procedure with the added benefit of a substantially larger production cross-section than Z+jets process used before.\\

The conclusions and results of each chapter are presented in the corresponding chapter.\\

This thesis work has been presented at several internal meetings of the CMS Experiment and at the following international meetings and conferences:

\begin{enumerate}
	\item{\textbf{Andr\'es Abreu} gave a talk ``\textbf{\textit{Estimation of the Z Invisible Background for Searches for Supersymmetry in the All-Hadronic Channel}}'' at ``APS April 2018: American Physical Society April Meeting 2018, 14-17 Apr 2018'', Columbus, OH}
	\item{\textbf{Andr\'es Abreu} gave a talk ``\textbf{\textit{Phase-2 Pixel upgrade simulations}}'' at the ``USLUA Annual meeting: 2017 US LHC Users Association Meeting, 1-3 Nov 2017'', Fermilab, Batavia, IL}
	\item{\textbf{Andr\'es Abreu} gave a talk, ``\textbf{\textit{Direct production of top squark pairs in all-hadronic channel}}'' presentation at  ``FNAL50: Fermilab 50th Anniversary Symposium and Users Meeting and New Perspectives  Workshop, Fermilab, Batavia, June 5-8, 2017'', Batavia, IL}
\end{enumerate}


