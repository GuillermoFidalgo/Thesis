
The work for this thesis was performed at CERN \cite{What_is_CERN} on CMS Experiment \cite{What_is_CMS}. CERN stands for European Organization for Nuclear Research. 
It was founded in 1954 and is located at the Franco-Swiss border near Geneva. At CERN, physicists and engineers are probing the fundamental structure of the universe. They use the world's largest and most complex scientific instruments to study the basic constituents of matter – the fundamental particles. The instruments used at CERN are purpose-built particle accelerators and detectors. Accelerators boost beams of particles to high energies before the beams are made to collide with each other or with stationary targets. Detectors observe and record the results of these collisions. The accelerator at CERN is called the Large Hadron Collider (LHC) \cite{What_is_LHC}, the largest machine ever built by humans and it collides particles (protons) at close to the speed of light. The process gives the physicists clues about how the particles interact, and provides insights into the fundamental laws of nature. Seven experiments at the LHC use detectors to analyze particles produced by proton-proton collisions. The biggest of these experiments, ATLAS\cite{What_is_ATLAS} and CMS, use general-purpose detectors designed to study the fundamental nature of matter and fundamental forces and to look for new physics or evidence of particles that are beyond the Standard Model \cite{What_is_SM} . Having two independently designed detectors is vital for cross-confirmation of any new discoveries made. The other two major detectors ALICE\cite{What_is_ALICE} and LHCb\cite{What_is_LHCb}, respectively, study a state of matter that was present just moments after the Big Bang and preponderance of matter than antimatter.  Each experiment does important research that is key to understanding the universe that surrounds and makes us.\\

\Cref{LHCCMS} presents a basic description of the Large Hadron Collider and CMS Detector\\

\Cref{DQMchapter} gives a brief description of what is Data Quality Monitoring and what it's importance for this experiment. \\

\Cref{ML} is dedicated to describing the idea of Machine Learning and how to implement this tool for this project.\\

\Cref{ch:results} Shows the results of this project.\\
