%Section 3.1
\section{What is Data Collection for CMS?}

During data taking there are millions of collisions occurring in the center of the detector every second. The data per event is around one million bytes (1 MB), that is produced at a rate of about 
600 million events per second \cite{datataking}, that’s about 600 MB/s. Keeping in mind that only certain events are considered “interesting” for analysis, the task of deciding what events to consider out of all the data collected is a two-stage process. 
First, the events are filtered down to 100 thousand events per second for digital reconstruction and then more specialized algorithms filter the data even more to around 100 $\sim$ 200 events per second that are found interesting.
For CMS there is a Data Acquisition System that records the raw data to what’s called a High-Level Trigger farm which is a room full of servers that are dedicated to processing and classify this raw data quickly. 
The data then gets sent to what’s known as the Tier-0 farm where the full processing and the first reconstruction of the data are done. \cite{cmscomputing} 


\section{What is Data Quality Monitoring?}
To operate a sophisticated and complex apparatus as CMS, a quick online feedback on the quality of the data recorded is needed to avoid taking low quality data and to guarantee a good baseline for the offline analysis. Collecting a good data sets from the collisions is an important step towards search for new physics as deluge of new data poses an extra challenge of processing and storage. This all makes it all the more important to design algorithms and special software to control the quality of the data. This is where the Data Quality Monitoring (DQM)\cite{1742-6596-219-7-072020} plays a critical in the maintainability of the experiment, the operation efficiency and performs a reliable data certification.  The high-level goal of the system is to discover and pinpoint errors, problems occurring in detector hardware or reconstruction software, early, with sufficient accuracy and clarity to maintain good detector and operation efficiency. The DQM workflow consists of 2 types: \textbf{Online}  and \textbf{Offline}.

The \textbf{Online} DQM consists of receiving data taken from the event and trigger histograms to produce results in the form of monitoring elements like histogram references and quality reports. This live monitoring of each detector’s status during data taking gives the online crew the possibility to identify problems with extremely low latency, minimizing the amount of data that would otherwise be unsuitable for physics analysis. The scrutiny of the Online DQM is a 24/7 job that consists of people or shifters that work at the CMS control center constantly monitoring the hundreds of different plots and histograms produced by the DQM software. This consumes a lot of manpower and is strenuous work.


The \textbf{Offline} DQM is more focused on the full statistics over the entire run of the experiment and works more on the data certification. In the offline environment, the system is used to review the results of the final data reconstruction on a run-by-run basis, serving as the basis for certified data used across the CMS collaboration in all physics analyses. In addition, the DQM framework is an integral part of the prompt calibration loop. This is a specialized workflow run before the data are reconstructed to compute and validate the most up-to-date set of conditions and calibrations subsequently used during the prompt reconstruction.

This project aims to minimize the DQM scrutiny by eye and automate the process so that there is a more efficient process to monitor the detector and the quality of the data by implementing Machine Learning techniques.