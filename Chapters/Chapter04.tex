
Machine Learning (ML) can be defined as an application of Artificial Intelligence that permits the computer system to learn without being told explicitly. 
In ML a computer program is said to learn from experience E with respect to some class of tasks T and performance measure P, if its performance at tasks in T, as measured by P, improves with experience E \cite{Coursera}.
 ML has made tremendous strides in the past decades and has become very popular recently due to its multifaceted applications. It is being used on social media, marketing, and in the scientific community as well. 
Some examples of ML applications are: the algorithms used on application in smartphones to detect human faces, self-driving cars, computer games, stock prediction, and voice recognition. An interesting characteristic of ML algorithms is that the more data one inputs the better is the performance. The ML application has a very wide spectrum covering almost every aspect of human endeavor that involves a lot of data. 
Scientific analysis today generates enormous data and is a hence is a perfect used case to apply ML techniques. In this work we use enhanced ML techniques based on progress in the recent past.

In general, there are two main categories to classify machine learning problems: \textbf{Supervised Learning} (SL) and \textbf{Unsupervised Learning} (UL). SL is the most used ML approach and has proven to be very effective for a wide variety of problems. Examples of common SL problems are: spam filters, predicting housing prices, identifying a malignant or benign tumor, etc. These types of problems are characterized by providing a “right answer” as a reference. For example, spam filter algorithms identify emails that are spams by training on a dataset that has examples of such emails. In case of predicting house prices, the algorithm is trained on a dataset of houses involving features like the area of the house, number of rooms, and the selling price of the house.

UL algorithms are different in the sense that they do not have the “right answers” given to the machine. Instead, UL algorithms are used for finding patterns and make clusters from the given data. That is what also forms the basis of a search engine (e.g. Google news). Clicking on a link to a news article, one gets many different stories of different journals that have some correlation with the article searched. This happens because the ML algorithm is capable of learning features and shared patterns from a bunch of data without being given any specifics. Another interesting UL problem is the so-called “cocktail party” that involves distinguishing the voice of two people recording on two microphones located at different places. The ML algorithm is able to separate the sources of the voices in the recordings by learning the voice features that correspond to each person, showing the power of the UL algorithm.

In this study, I have focused on an SL approach and a variant of the UL approach, called the \textbf{Semi-Supervised Learning} approach (SSL). The SSL is named so because the data involves looking at images that are already known to be “Good” but one doesn’t necessarily know every possible situation that produces a “Bad” image. The purpose is to define a metric for a “good” image and subsequently decide if an image is “bad” in case it deviates too much from an acceptable value.



\section{Developing the Algorithm}

To develop an ML algorithm the following are taken into consideration, what is the task? and what is the method to approach the task? In our case, we are looking into images that have information about the activity that the channels in the HCAL are detecting. These images are called "occupancy maps" and they are a visual way of monitoring the health of the detector itself (See Figure 3). There are two common problems that can be identified by viewing occupancy maps which are called "dead channels" and "hot towers". They are referred to as “dead” and “hot” respectively in the rest of this document. Dead channels mean that on a certain place in the occupancy map there is not any readout from the channels on the HCAL and hot channels mean that there are channels that are being triggered by noise or are damaged in a way that makes them readout too much activity.

\begin{center}
\begin{figure}
\begin{subfigure}[h]{0.20\linewidth}
\includegraphics[width=\linewidth]{Good_image.jpg}
\caption{Good Image\label{Goodimage}}
\end{subfigure}
\begin{subfigure}[h]{0.30\linewidth}
\includegraphics[width=\textwidth]{Dead_image_5x5.jpg}
\end{subfigure}
\end{figure}
\end{center}