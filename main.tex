%Specify document class
\documentclass[12pt,twoside,openany]{book}



%Packages to include
\usepackage[utf8]{inputenc}
\usepackage{setspace}
\usepackage{times}
\usepackage[a4paper,width=150mm,top=30mm,bottom=1in,bindingoffset=6mm]{geometry}
\usepackage{graphicx,wrapfig,lipsum}
\usepackage{bm}
\usepackage{indentfirst}
\usepackage{verbatim}
\usepackage{float}
\usepackage[shortlabels]{enumitem}
\usepackage{subcaption}
\usepackage{titlesec}
\usepackage[font={footnotesize}]{caption}
\usepackage{amsmath}
\usepackage{amssymb}


%\usepackage[subfigure,titles]{tocloft}
\usepackage{cite}
\usepackage{url}
\usepackage{xfrac}

\usepackage[nottoc,numbib]{tocbibind}
\usepackage[table,xcdraw]{xcolor}
\usepackage{siunitx}
\usepackage{lineno}
\settocbibname{References}
\usepackage[%  
    colorlinks=true,
    pdfborder={0 0 0},
    linkcolor=blue,
    citecolor=blue,
    urlcolor=blue
]{hyperref}
\usepackage[nameinlink]{cleveref}

\linenumbers


%Set images folder
\graphicspath{ {images/} }

%Set Header and Footer for all pages
\usepackage{fancyhdr}
%\bibliographystyle{ieeetr}

\pagestyle{fancy}
\fancyhead{}
\fancyhead[LO]{\nouppercase{\textbf{\leftmark}}}
\fancyhead[RO,LE]{\textbf{\thepage}}
\fancyhead[RE]{\nouppercase{\textbf{\rightmark}}}
\fancyfoot{}
\renewcommand{\chaptermark}[1]{%
\markboth{#1}{}}

%Set Title Page Background
\usepackage[pages=some]{background}
\backgroundsetup{
scale=1,
color=black,
opacity=0.1,
angle=0,
contents={\includegraphics[width=\textwidth]{university.png}}
}

\setlist[itemize]{topsep=\parskip}

%\setlength\abovecaptionskip{-5ex}
%\setlength{\textfloatsep}{0pt plus 1.0pt minus 2.0pt}
\setlength\belowcaptionskip{-3ex}

\usepackage[labelfont=bf]{caption}
\captionsetup{labelfont=bf}

%Eliminate extra page after title and table of contents
\let\cleardoublepage=\clearpage

%Make signature and date lines for title page
\newcommand*{\SignatureAndDate}[1]{
    \par\noindent\makebox[3.0in]{\hrulefill} \hfill\makebox[2.0in]{\hrulefill}
    \par\noindent\makebox[2.5in][l]{#1}      \hfill\makebox[2.0in][l]{Date}
}

%for list of figures and tables
\newlength{\mylen}
%
%\renewcommand{\cftfigpresnum}{\figurename\enspace}
%\renewcommand{\cftfigaftersnum}{:}
%\settowidth{\mylen}{\cftfigpresnum\cftfigaftersnum}
%\addtolength{\cftfignumwidth}{\mylen}

%Main Body
\begin{document}

%Title Page
\begin{titlepage}
    \begin{center}
    
      % \ThisCenterWallPaper{1}{university.png}
       \BgThispage
        \vspace*{1cm}
      \Large  
        \textbf{Measurement of the Z Invisible background for Stop Quark Searches in all Hadronic Channel at \bm{$\sqrt{s} = 13$} TeV}
        
        \normalsize
        \vspace{0.5cm}
        by
        
        \vspace{0.25cm}
        
        Andr\'es J. Abreu Nazario
        
        \vspace{0.5cm}	
        
        A thesis presented for the degree of\\
        \vspace{0.5cm}
        MASTER OF SCIENCE\\
        \vspace{0.5cm}
        in\\
        \vspace{0.5cm}
        Physics\\
        \vspace{0.5cm}
        UNIVERSITY OF PUERTO RICO\\
        MAYAG\"UEZ CAMPUS\\
        2018
        \vspace{0.8cm}
        
        \end{center}
        
        Approved by:
        
        \vspace{1.0cm}

\SignatureAndDate{Sudhir Malik, Ph.D.}
\\President, Graduate Committee

\vspace{1.0cm}

\SignatureAndDate{H\'ector M\'endez, Ph.D.}
\\Member, Graduate Committee

\vspace{1.0cm}

\SignatureAndDate{Samuel Santana Col\'on, Ph.D.}
\\Member, Graduate Committee

\vspace{1.0cm}

\SignatureAndDate{Rafael A. R\'amos, Ph.D.}
\\Chairperson of the Department



\end{titlepage}

\frontmatter

\doublespace
\chapter{Abstract}
The Data Quality Monitoring (DQM) of CMS is a key asset to deliver high-quality data for physics analysis and it is used both in the online and offline environment. The current paradigm of the quality assessment is labor intensive and it is based on the scrutiny of a large number of histograms by detector experts comparing them with a reference. This project aims at applying recent progress in Machine Learning techniques to the automation of the DQM scrutiny. In particular the use of convolutional neural networks to spot problems in the acquired data is presented with particular attention to semi-supervised models (e.g. autoencoders) to define a classification strategy that doesn’t assume previous knowledge of failure modes. Real data from the hadron calorimeter of CMS are used to demonstrate the effectiveness of the proposed approach. 

\vspace*{1cm}

\textit{Keywords}:  [DQM, online, offline, Machine Learning ]

\chapter{Acknowledgments}

I wish to thank United States State Department and University of Michigan for providing the opportunity to work abroad at CERN during the 2018 Winter Semester. I also wish to thank CERN staff, CMS Experiment , Texas Tech University, and the University of Puerto Rico at Mayagüez, with special thanks to Dr. Federico de Guio for his local mentorship and Dr. Nural Akchurin, and Dr. Sudhir Malik for their guidance. A very special thanks to Dr. Jean Krisch for accepting me for this great research opportunity and Dr. Steven Goldfarb for being a wonderful host and overall local guidance at CERN.
\singlespace

\listoffigures

\tableofcontents


\doublespace
\mainmatter
\chapter{Introduction}

The work for this thesis was performed at CERN \cite{What_is_CERN} on CMS Experiment \cite{What_is_CMS}. CERN stands for European Organization for Nuclear Research. 
It was founded in 1954 and is located at the Franco-Swiss border near Geneva. At CERN, physicists and engineers are probing the fundamental structure of the universe. They use the world's largest and most complex scientific instruments to study the basic constituents of matter – the fundamental particles. The instruments used at CERN are purpose-built particle accelerators and detectors. Accelerators boost beams of particles to high energies before the beams are made to collide with each other or with stationary targets. Detectors observe and record the results of these collisions. The accelerator at CERN is called the Large Hadron Collider (LHC) \cite{What_is_LHC}, the largest machine ever built by humans and it collides particles (protons) at close to the speed of light. The process gives the physicists clues about how the particles interact, and provides insights into the fundamental laws of nature. Seven experiments at the LHC use detectors to analyze particles produced by proton-proton collisions. The biggest of these experiments, ATLAS\cite{What_is_ATLAS} and CMS, use general-purpose detectors designed to study the fundamental nature of matter and fundamental forces and to look for new physics or evidence of particles that are beyond the Standard Model \cite{What_is_SM} . Having two independently designed detectors is vital for cross-confirmation of any new discoveries made. The other two major detectors ALICE\cite{What_is_ALICE} and LHCb\cite{What_is_LHCb}, respectively, study a state of matter that was present just moments after the Big Bang and preponderance of matter than antimatter.  Each experiment does important research that is key to understanding the universe that surrounds and makes us.\\

\Cref{LHCCMS} presents a basic description of the Large Hadron Collider and CMS Detector\\

\Cref{DQMchapter} gives a brief description of what is Data Quality Monitoring and what it's importance for this experiment. \\

\Cref{ML} is dedicated to describing the idea of Machine Learning and how to implement this tool for this project.\\

\Cref{ch:results} Shows the results of this project.\\


\chapter{The CMS Experiment\label{LHCCMS}}
%\section{The CMS Experiment}

The Compact Muon Solenoid (CMS) detector is a general purpose particle detector designed to investigate various physical phenomena concerning the SM and beyond it, such as Supersymmetry \cite{What_is_SUSY} ,
 Extra Dimensions and Dark Matter \cite{What_is_DM} . As its name implies, the detector is a solenoid which is  constructed around a superconducting magnet capable of producing a magnetic field of 3.8 T. The magnetic coil is 13m long with an inner diameter of 6m, making it the largest superconducting magnet ever constructed. The CMS detector itself is 21m long with a diameter of 15m and it has a weight of approximately 14,000 tons. The CMS experiment is one of the largest scientific collaborations in the history of mankind with over 4,000 participants from 42 countries and 182 institutions. CMS is located at one of these points and it essentially acts as a giant super highspeed camera that makes 3D images of the collisions that are produced at a rate of 40 MHz (40 million times per second). The detector has an onion-like structure to capture all the particles that are produced in these high energy collisions most of them being unstable and decaying further to stable particles that are detected.  CMS detector was designed with the following features (as shown in \autoref{CMSLayout}) :

\begin{enumerate}
	\item{A \textbf{magnet} with large bending power and high performance muon detector for good muon
identification and momentum resolution over a wide range of momenta and angles.}

	\item{An \textbf{inner tracking system} capable of high reconstruction efficiency and momentum resolution
requiring \textbf{pixel detectors} close to the interaction region.}

	\item{An \textbf{electromagnetic calorimeter} able to provide good electromagnetic energy resolution and  
a high isolation efficiency for photons and leptons.}

	\item{A \textbf{hadron calorimeter} capable of providing precise missing-transverse-energy and dijet-mass  
resolution.}

\end{enumerate}

\begin{figure}
\includegraphics[width=\linewidth]{CMSLayout.png}
\caption{CMS Detector \label{CMSLayout}}
\end{figure}

A property from these particles that is exploited is their charge. Normally, particles produced in collisions travel in a straight line, but in the presence of a magnetic field, their paths are skewed and curved. Except the muon system, the rest of the subdetectors lie inside a 3.8 Tesla magnetic field . Due to the magnetic field the trajectory of charged particle produced in the collisions gets curved  (as shown in \autoref{CMSLayers} ) and one can calculate the particle’s momentum and know the type of charge on the particle.  The Tracking devices are responsible for drawing the trajectory of the particles by using a computer program that reconstructs the path by using electrical signals that are left by the particle as they move.  The Calorimeters measure the energy of particles that pass through them by absorbing their energy with the intent of stopping them. The particle identification detectors work by detecting radiation emitted by charged particles and using this information they can measure the speed, momentum, and mass of a particle. After the information is put together to make the “snapshot” of the collision one looks for results that do not fit the current theories or models in order to look for new physics.

\begin{figure}[h]
\includegraphics[width=.90\linewidth]{CMSLayers.PNG}
\caption{The trajectory of a particle traveling through the layers of the detector leaving behind it's signature footprint\label{CMSLayers}}
\end{figure}


The project focusses specifically on data collected from one of the Calorimeters, - the Hadron Calorimeter (HCAL). The HCAL, as its name indicates, is designed to detect and measure the energy of hadrons or, particles that are composed of quarks and gluons, like protons and neutrons. Additionally, it provides an indirect measurement of the presence of non-interacting, uncharged particles such as neutrinos (missing energy) . Measuring these particles is important as they can tell us if new particles such as the Higgs boson or supersymmetric particles (much heavier versions of the standard particles we know) have been formed. The layers of the HCAL are structured in a staggered fashion to prevent any gaps that a particle might pass through undetected. There are two main parts: the barrel and the end caps. There are 36 barrel wedges that form the last layer of the detector inside the magnet coil, there is another layer outside this, and on the endcaps, there are another 36 wedges to detect particles that come out at shallow angles with respect to the beam line.




\chapter{Data Collection and Data Quality Monitoring \label{DQMchapter} }
%Section 3.1
\section{What is Data Collection for CMS?}

During data taking there are millions of collisions occurring in the center of the detector every second. The data per event is around one million bytes (1 MB), that is produced at a rate of about 600 million events per second (CERN, 2018), that’s about 600 MB/s. Keeping in mind that only certain events are considered “interesting” for analysis, the task of deciding what events to consider out of all the data collected is a two-stage process. First, the events are filtered down to 100 thousand events per second for digital reconstruction and then more specialized algorithms filter the data even more to around 100~200 events per second that are found interesting.
For CMS there is a Data Acquisition System that records the raw data to what’s called a High-Level Trigger farm which is a room full of servers that are dedicated to processing and classify this raw data quickly. The data then gets sent to what’s known as the Tier-0 farm where the full processing and the first reconstruction of the data are done. \cite{cmscomputing} 


%\begin{comment} %
\begin{figure}[tb]
\begin{center}
\includegraphics[width=1.0\textwidth]{Cern-Accelerator-Complex.jpg} 
\caption{The CERN Accelerator Complex\cite{LHCcomplex}.}
\label{Cern-Accelerator-Complex.jpg} 
\hspace{4em}
\end{center}
\end{figure}
%\end{comment} %

The four main detectors comprising the LHC machine are CMS, ATLAS\cite{ATLAS}, LHCb\cite{LHCb} and ALICE\cite{ALICE}. Both CMS and ATLAS are general purpose detectors whose initial designs had the detection of the SM Higgs boson, with its wide range of decay modes, in mind. Both detectors managed to accomplish this goal when a 126 GeV scalar boson consistent with the SM Higgs was independently verified by both experiments in July of 2012. Furthermore, the designs for CMS and ATLAS allow for the search of many other additional phenomena in BSM physics such as Supersymmetry, Dark Matter\cite{DM}, Dark Sector\cite{DS}, etc. On the other hand, the LHCb and ALICE detectors focus on more particular kinds of searches. The main motivation for the LHCb experiment, where the b stands for beauty, concerns itself with the measurement of CP violation parameters in b-hadron interactions and studies cover a wide range of aspects of Heavy Flavor Electroweak and QCD physics. Meanwhile, the ALICE experiment focuses on the study of heavy ion (Pb-Pb) nuclei collisions at a centre-of-mass energy of 2.76 TeV in order to better understand the physics behind strongly interacting matter at extreme energy densities.

%Section 3.2
\section{The CMS Detector}
The Compact Muon Solenoid (CMS) detector is a general purpose particle detector designed to investigate various physical phenomena concerning the SM and beyond it, such as Supersymmetry, Extra Dimensions and Dark Matter. As its name implies, the detector is a solenoid which is constructed around a superconducting magnet capable of producing a magnetic field of 3.8 T. The magnetic coil is 13m long with an inner diameter of 6m, making it the largest superconducting magnet ever constructed. The CMS detector itself is 21m long with a diameter of 15m and it has a weight of approximately 14,000 tonnes. The CMS experiment is one of the largest scientific collaborations in the history of mankind with over 4,000 participants from 42 countries and 182 institutions.\\

%\begin{comment} %
\begin{figure}[tb]
\includegraphics[width=1.0\textwidth]{CMSLayout.png} 
\caption{The CMS Detector Layout\cite{CMSlayout}.}
\label{CMSLayout} 
\hspace{4em}
\end{figure}
%\end{comment} %

In order to meet the many needs of the SM and BSM searches, and the goals of the LHC physics program, the CMS detector was designed with the following features:\\
\begin{itemize}
	\itemsep-1em
	\item{A magnet with large bending power and high performance muon detector for good muon identification and momentum resolution over a wide range of momenta and angles.}\\
	\item{An inner tracking system capable of high reconstruction efficiency and momentum resolution requiring pixel detectors close to the interaction region.}\\
	\item{An electromagnetic calorimeter able to provide good electromagnetic energy resolution and a high isolation efficiency for photons and leptons.}\\
	\item{A hadron calorimeter capable of providing precise missing-transverse-energy ($p_\text{T}^{miss}$)  and dijet-mass resolution.}\\
\end{itemize}

A general layout of the CMS detector and all its constituent sub-detectors can be seen in \autoref{CMSLayout}. The configuration of the CMS sub-detectors follow a cylindrical layer pattern that is symmetrical about the interaction region and consists of a central barrel with endcaps on both ends.

The coordinate system for the CMS detector design uses a right-hand rule convention centered around the ideal interaction point to describe the positions of objects in the experiment. The z-axis is defined along the direction of the LHC beam, with the x-axis pointing towards the center of the LHC ring. In terms of polar coordinates then, r is the radial distance from the center of the pipe, the polar angle $\theta$ is measured against the z-axis and the azimuthal angle $\phi$ is measured with respect to the x-axis. However, the pseudorapidity $\eta$ is generally preferred over the polar angle $\theta$. The pseudorapidity is defined as:

\begin{center}
$\eta$ =  $-$ ln tan $\frac{\theta}{2}$.
\end{center}

\subsection{Silicon Tracking System}

The CMS tracking system was designed with the goal of obtaining precise and efficient measurements for the trajectories of charged particles resulting from proton-proton collisions at the LHC. 
In addition, it allows for the precise measurement of secondary vertices and impact parameters needed to efficiently identify the heavy flavours produced in many interesting physics channels. Due to the LHC's design Luminosity of 10$^{34}$ cm$^{-2}$s$^{-1}$, the currently installed CMS phase-1 tracker is expected to handle an average of 1000 particles from over 20 overlapping proton-proton interactions per bunch crossing, every 25 ns. This required a detector technology capable of achieving a high granularity and fast response as well as being tolerant to the radiation produced from the intense particle flux. These considerations lead to a tracker design composed entirely of silicon detector technology which features an active silicon area of about 200 m$^2$, making it the largest silicon tracker ever built\cite{CMSdet1}.\\

%\begin{comment} %
\begin{figure}[tb]
\begin{center}
\includegraphics[width=0.9\textwidth]{CMSTrackerLayout.png} 
\caption{Overview of the CMS Tracker Layout\cite{CMSdet2}.}
\label{CMSTrackerLayout} 
%\hspace{4em}
\end{center}
\end{figure}
%\end{comment} %

The CMS tracker is built in a cylindrical manner around the interaction point and has a diameter of 2.5 m and a length of 5.8 m. It is comprised of a pixel detector with three barrel layers, positioned at a distance between 4.4 cm and 10.2 cm from the interaction region, and a silicon strip tracker with 10 barrel detection layers extending to a radius of 1.1 m. Each system is made complete by endcaps at opposite sides of the barrel, consisting of 2 discs for the pixel detector and 3 plus 9 discs for the strip tracker, extending the acceptance region of the tracker up to a pseudorapidity $|\eta| <$ 2.5.\\

The pixel detector is the part of the CMS tracking system closest to the interaction region and consists of 3 barrel layers (BPix) and 2 endcap discs (FPix). It is responsible for providing precise tracking points in $r$-$\phi$ and $z$, a feature that is required for the small impact parameter resolution, needed for good secondary vertex reconstruction. The detector contains 1440 modules covering an area of approximately 1 m$^2$ making up a total of 66 million pixels. The sensors were designed using an n-on-n concept from 320 $\mu$m thick silicon wafers and are fabricated with read-out chips (ROCs) that are bump-bonded to the sensor in standard 0.25 $\mu$m CMOS technology. Each of the pixels has a pitch size of 100 $\times$ 150 $\mu$m$^2$, which corresponds to an occupancy of about 10$^{-4}$ per bunch crossing.\\

The silicon strip tracker is built surrounding the pixel detector and consists of three large subsystems. The Tracker Inner Barrel and Disks (TIB/TID) extend to a radius of 55 cm and are composed of four barrel layers, completed by three disks at each end. The TIB/TID employs the use of 320 $\mu$m thick silicon micro-strip sensors in order to deliver up to 4 $r$-$\phi$ measurements on a trajectory. Surrounding the TIB/TID is the Tracker Outer Barrel (TOB), which consists of 6 barrel layers and has an outer radius of 116 cm. The TOB extends symmetrically in $z$ between $\pm$118 cm and provides an additional 6 $r$-$\phi$ measurements for a trajectory. Beyond the TOB's $z$ range lie the Tracker Endcaps (TEC+ and TEC-, where the sign indicates their position respect to $z$). Each TEC consists of 9 discs with up to 7 rings of silicon micro-strip detectors, providing up to 9 additional $\phi$ measurements per trajectory. The CMS silicon strip tracker has a total active silicon area of 198 m$^2$ and is composed of 15,148 sensor modules.

\subsection{Electromagnetic Calorimeter}
The CMS Electromagnetic Calorimeter (ECAL) is a hermetic homogeneous calorimeter whose function is to measure the energy of particles that interact via the electromagnetic force. With the use of 75,848 scintillator crystals, it is capable of providing good energy resolution within the requirements of the ambitious LHC program. In particular, the ECAL's design was optimized to search for diphoton events resulting from Higgs boson decays (H $\rightarrow \gamma \gamma$).\\ 

%\begin{comment} %
\begin{figure}[tb]
\begin{center}
\includegraphics[width=0.8\textwidth]{ECALgeometry.png} 
\caption{Geometrical layout of the CMS Electromagnetic Calorimeter\cite{CMSLHC}.}
\label{CMS_ECAL_Geometry} 
\hspace{4em}
\end{center}
\end{figure}
%\end{comment} %

The ECAL is composed of two main sub-systems -- the barrel calorimeter (EB) and the endcap calorimeter (EE) -- and is completed by a preshower calorimeter (ES), as shown in \autoref{CMS_ECAL_Layout}. It covers a solid angle up to a pseudorapidity of $|\eta| <$ 3, with the EB extending in the range of $|\eta| <$ 1.479 and the EE covering a range of  1.479 $< |\eta| <$ 3. Both subsystems are composed of lead tungstate (PbWO$_4$) scintillator crystals which provide a fast response time and high radiation tolerance, both crucial requirements for optimal performance at LHC operating conditions. In addition, the properties of the PbWO$_4$ crystals (high density, short radiation and small Moli\'ere radius) led to the design of a compact calorimeter with fine granularity.\\

The barrel part of the ECAL (EB) is composed of specially designed avalanche photodiodes (APD). It consists of 61,200 crystals, forming a total volume of 8.14 m$^3$ and weighing about 67.4 t. The crystals that form the EB are 230 mm long with a cross-section of 22$\times$22 mm$^2$ at the front face and 26$\times$26 mm$^2$ at the back. They are organized in pairs within thin-walled alveolar structures called submodules. The submodules are assembled into different types of modules that differ by their location with respect to $\eta$ and contain 400 or 500 crystals each. Theses modules are then arranged in sets of four modules, called supermodules, and contain 1700 crystals each. Thus, the EB is composed of two half-barrels, each consisting of 18 supermodules.\\

In contrast, the photodetectors used in the endcap section of the ECAL are vacuum phototriodes (VPT)\cite{VPT}. Each of the endcaps contain 7,324 crystals which, in total, occupy a volume of 2.90 m$^3$ and weigh about 24.0 t. The crystals in the EE are 220 mm in length with a cross-section of 28.62$\times$28.62 mm$^2$ for the front face and 30$\times$30 mm$^2$ in the rear. They are all identical in shape and are arranged in mechanical units of 5$\times$5 crystals, called supercrystals (SC), which consist of carbon-fibre alveola structures. Each of the endcaps are divided into two semi-circular structures, called \textit{Dees}, which hold a total of 3,662 crystals. \\

The ES preshower is located before the EE detector and spans a pseudorapidity range of 1.653 $< |\eta| <$ 2.6. It's main purpose is to identify neutral pions in the endcaps as well as to improve the determination of electrons and photons with high granularity. The preshower consists of two layers and has a total thickness of 20 cm. The first layer is conformed by lead radiators which initiate electromagnetic showers from incoming photons and electrons. Meanwhile, the second layer is composed of silicon strips which are capable of measuring the deposited energy and the transverse shower profiles.

%\begin{comment} %
\begin{figure}[tb]
\begin{center}
\includegraphics[width=0.9\textwidth]{ECALlayout.png} 
\caption{Layout of the CMS ECAL illustrating its various components\cite{CMSdet1}.}
\label{CMS_ECAL_Layout}
\vspace{-1em}
\end{center}
\end{figure}
%\end{comment} %

\subsection{Hadron Calorimeter}

The CMS Hadron calorimeter (HCAL) conforms the next layer of the CMS detector. It is a sampling calorimeter that consists of alternating layers of massive absorbing brass plates and plastic scintillator tiles and is of particular importance for the measurement of hadron jet energy and  $p_\text{T}^{miss}$. The HCAL detector is located in between the outer extent of the ECAL ($R$ = 1.77 m) and the inner extent of the magnet coil ($R$ = 2.95 m). Similar to the other CMS subsystems, it's composed of a barrel part (HB) and an endcap part (HE). In addition, it features a tail-catching outer calorimeter (HO), located outside the magnet, and a forward calorimeter (HF) in the very forward region near the beam line. A layout of the HCAL system can be seen in \autoref{CMS_HCAL_Layout}.\\

%\begin{comment} %
\begin{figure}[tb]
\begin{center}
\includegraphics[width=0.9\textwidth]{CMS_HCAL.png} 
\caption{Geometrical layout of the HCAL showing the locations of the hadron barrel (HB), endcap (HE), outer (HO) and forward (HF) calorimeters\cite{CMShcal}.}
\label{CMS_HCAL_Layout} 
\end{center}
\end{figure}
%\end{comment} %

The barrel component of the HCAL is a sampling calorimeter which covers the pseudorapidity range $|\eta|<$1.3. It consists of both the HB and the HO detectors. The reason behind separating the barrel detector into the HB and HO is due to the limited amount of space available for the barrel detector. The HB is located within the superconducting magnet coil and is supplemented by the HO in between the outer solenoid coil and the muon chambers. Therefore, the HO acts as a tail-catcher in order to improve the jet energy measurements and $p_\text{T}^{miss}$, with the solenoid in between acting as absorber material. The HB consists of two half-barrel sections, identified as HB+ and HB- due to their geometrical location, which are composed of 36 identical azimuthal wedges. The wedges, which are constructed out of flat brass absorber plates, are aligned parallel to the beam axis and are segmented into four azimuthal angle ($\phi$) sections.\\ 

The HE covers a significant amount of the pseudorapidity in the range of 1.3 $< |\eta| <$ 3, a region containing about 34\% of the particles produced in the final state. Due to the high luminosity of the LHC, the HE is required to have a high radiation tolerance at $|\eta| \simeq 3$, as well as being capable of handling high counting rates. Similar to the HB, the HE is also composed of brass absorber plates and scintillator plates which are read out by wavelength shifting fibers. The light captured by the scintillators merges within the wavelength shifting fibers and then it's read out by hybrid photo-diodes. The scintillators are partitioned in towers with an area of $\Delta\eta\times\Delta\phi = 0.17\times0.17$.\\

\subsection{Magnet}

The CMS superconducting solenoid is one of the driving features of the detector design. It is capable of providing a magnetic field with a 3.8T magnitude, which allows for the large bending power needed for  precise particle transverse momentum ($p_{\text{T}}$) measurements. The magnet is made of four layers of stabilized reinforced Niobium-Titanium (NbTi) and has a cold mass of 220 t. The solenoid consists of a 13 m long coil with an internal diameter of 6 m, which houses both the tracking and calorimetric system. This design allows for particles to be measured prior to crossing the magnetic coil which significantly improves the energy resolution. 

\subsection{Muon Detector}

 As implied by the detector's name, precise and robust muon measurements have been a central theme of the CMS experiment since the early stages of its design. The detector design takes into account that muons behave as minimum ionizing particles (MIPs)\cite{MIPs} and can therefore manage to traverse
the tracker and calorimeters with minimal energy loss. Furthermore, due to their relatively long lifetime they can be efficiently identified by a dedicated system at the outer region of the detector. Consequently, the CMS muon systems comprise the outermost layer of the detector, which are integrated into the magnet return yoke that surrounds the solenoid.\\

The muon system is capable of three main functions: muon identification, muon $p_{\text{T}}$ measurement and triggering. It is composed of three different types of detectors, all of which make use of gaseous chamber technology. This choice of detector provides a cost efficient way of covering most of the full solid angle, featuring a total of 25,000 m$^2$ of detection plates. As a consequence of the shape of the solenoid magnet, the muon detector was designed to have a cylindrical barrel section as well as two planar endcap regions.\\

\begin{figure}[H]
\begin{center}
\begin{minipage}[b]{0.59\textwidth}
\includegraphics[width=\textwidth]{muonAlignment.png}
\end{minipage}
\hspace{1em}
\begin{minipage}[b]{0.3\textwidth}
    \includegraphics[width=\textwidth]{MuStations.png}
\end{minipage}
\end{center}
\vspace{-1em}
\caption[Quarter-view of the CMS muon system (left) and a muon in the transverse plane leaves a curved trajectory across the four muon detector stations (right).]{The left diagram shows a quarter-view of CMS with both the muon barrel (MB) and endcap (ME) stations \cite{muonAlignment}.  The right diagram shows a muon in the transverse plane leaves a curved trajectory across the four muon detector stations\cite{MuDet}.}
\label{muonSystem}
\end{figure}

For the barrel region of the muon detector, four layers of drift tube (DT) modules are used, covering a pseudorapidity of up to $|\eta| < 1.2$. These four layers, called ``stations'', are arranged in cylindrical concentric layers around the beam line, where the first three layers have 60 DTs each and the outer cylinder has 70. Each of the DT stations contain 12 individual gas filled tubes, all of which have a 4cm diameter and a center electrode. The use of DTs as tracking detectors for the barrel muon system is possible because of the low expected muon rate and the relatively low strength of the local magnetic field.\\
 
The endcap regions of the muon system are subject to a higher muon rate, and cover the range $0.9 < |\eta| < 2.4$ where the magnetic field is stronger and less homogeneous. Considering these conditions, cathode strip chambers (CSCs) are employed, which feature high granularity, fast response time and adequate radiation hardness. CSCs consist of arrays of positively-charged ``anode'' wires crossed with negatively-charged copper ``cathode'' strips within a gas volume. They are trapezoidal in shape and can cover either $\ang{10}$ or $\ang{20}$ in $\phi$. Furthermore, CSCs have the advantage of featuring both precision muon measurement and muon trigger in a single device.\\

The third type of detector used in the CMS muon system are called resistive plate chambers (RPCs) and can be found in both the barrel and endcap regions. They consist of gaseous parallel-plate detectors capable of providing precise timing information and adequate spatial resolution. Because of their excellent time resolution, RPCs provide the capability of tagging the time of an ionizing event between 2 consecutive LHC bunch crossings (BX) in a much shorter time ($\sim$ 1 ns) than the interval between the BXs (25 ns). For this reason, an RPC-based dedicated muon trigger device can be implemented to unambiguously identify the relevant BX to which a muon track is associated with, despite the high rate of events and background expected at the LHC.

\subsection{Trigger and Data Acquisition}

Due to the vast volume of data originating from the proton-proton collisions (delivered by the LHC at a rate of 40 MHz), a method of eliminating the majority of the uninteresting/unwanted events was a requisite for the CMS detector design. This event rate reduction is achieved by the implementation of the so-called trigger system, which manages to select the potentially interesting interactions and reduce the rate from a staggering 40 TBs$^{-1}$ to a manageable value of just a few hundred Hz.\\

The CMS trigger system is implemented using a two-stage rate reduction, which combines both a hardware and software phase. The combination of both of these triggers is designed to reduce the rate by a factor of $\sim10^6$. The first stage used in the rate reduction is purely hardware based and it is called the Level 1 (L1) Trigger \cite{L1T}, which consists of both Field Programmable Gate Array (FPGA) and Application Specific Integrated Circuit (ASIC) technology. During this initial stage, the rate is reduced to about 100 kHz with a latency of 3.2 ${\mu}$s. This time interval constrains the trigger decision, allowing only for data from the calorimeters and muon system to be processed. Trigger primitives (TP) from these subsystems are processed through a series of steps before the combined event information is evaluated by the global trigger (GT) where the final decision, whether or not to accept the event, is made.\\

The second stage, which implements offline-quality reconstruction algorithms in its event selection, is referred to as the High-Level Trigger (HLT)\cite{HLT}. The event selection process for the HLT requires that physics objects for each event, such as electrons, muons and jets, are reconstructed and undergo predetermined identification criteria. 

\subsection{Event Reconstruction}

In order to reconstruct the events the particle flow (PF) algorithm\cite{PFref} is used. This algorithm gathers information from all the CMS sub-detectors to reconstruct charged and neutral hadrons, photons, muons, and electrons. It relies on an efficient and pure track reconstruction, a clustering algorithm able to distinguish overlapping tracks originated from different vertices, and an efficient link procedure to associate each particle deposit in the sub-detectors. Once all the deposits of a particle are associated, it can be correctly identified and its four-momentum optimally determined from the combined information of the sub-detectors. The resulting list of particles are then used to reconstruct higher level objects such as jets, taus, missing transverse energy, and to compute charged lepton and photon isolation, etc\cite{Reco1}. The CMS experiment is provided with millions of collisions per second, which need to be triggered, detected, stored and analyzed in a collaboration of several thousand physicists. This huge amount of data and the complexity of the detector require a flexible data model that serves all the needs of the collaboration. The data format is optimized for performance and flexibility of the reconstruction for the end user's analysis.\\

\begin{figure}[tb]
\begin{center}
%\includegraphics[width=1.0\textwidth]{CMSParticleDet.png} 
\caption{Transverse slice of the CMS detector, showing the individual detector subsystems and particle signatures in each. The particle type can be inferred by combining the detector response in the different subdetectors\cite{CMSslice}.}
\label{CMSParticleDet} 
\end{center}
\end{figure}

Event information from each step in the simulation and reconstruction chain is logically grouped into what is called a data tier\cite{CMScomp1}. From the physicist's point of view the most important data tiers are RECO, which contains all reconstructed objects and hits, and AOD (a subset of RECO). The AOD will contain a copy of all the high-level physics objects (such as muons, electrons, taus, etc.) and enough information about the event to support all the typical usage patterns of a physics analysis. It also contains a summary of the RECO information sufficient to support typical analysis actions such as track refitting with improved alignment or kinematic constraints, re-evaluation of energy and/or position of ECAL clusters based on analysis-specific corrections. The format of each data tier is ROOT \cite{ROOT}. ROOT is a framework for data processing developed at CERN with the sole purpose of aiding high energy physics research. The various AOD datasets are stored worldwide at various data tier centers. From the AOD's the analysis groups create data structures called NTuples containing only the high-level physics objects needed for their particular analysis. 

\subsection{Future Upgrade of Pixel Detector}
After the first LHC shutdown called LS1 (2013-2014), and the installation of the Phase-1 Pixel Detector \cite{Phase1FPix} in early 2017, among other things, the LHC is planning another series of upgrades during two major shutdowns, called LS2 and LS3, currently planned for 2019-2020 and $\sim$ 2024, respectively. LS2 would result in a further increase of the luminosity beyond the original design value, to over $2\cdot{10}^{34}$ cm$^{-2}$ s$^{-1}$.  With the LS3 upgrade of the LHC (called HL-LHC\cite{HLLHC}) the luminosity is expected to reach up to $7.5\cdot{10}^{34}$ cm$^{-2}$ s$^{-1}$. Correspondingly, the CMS collaboration has planned a series of further upgrades \cite{CMSPhase2,CMSupgrades} that will ensure the capabilities of the CMS detector to match to the HL-LHC running conditions, while taking the opportunity to improve the performance and repair any problems uncovered during the data-taking periods. The UPRM group will continue its involvement in the Phase-1 Pixel Detector operations and Phase-2 Pixel Upgrade design.\\

The HL-LHC conditions of instantaneous peak luminosities of up to $7.5\cdot{10}{^34}$ cm$^{-2}$ s$^{-1}$ and an integrated luminosity of the order of 300 fb$^{-1}$ would result in 1 MeV neutron equivalent fluence of $2.3\cdot{10}^{16}$ neq/cm$^2$ and a total ionizing dose (TID) of 12MGy (1.2 Grad) at the center of CMS, where its innermost component, the Phase-2 Pixel Detector will be installed. The detector should be able to withstand the above radiation dose, handle projected hit rates of 3GHz/cm$^2$ at the lowest radius, be able to separate and identify particles in extremely dense collision debris, deal with a pileup of 140-200 collisions per bunch crossing and have a high impact parameter resolution. This translates into requiring a detector design that is highly granular, has thinner sensors and smaller pixels, and faster, radiation hard electronics compared to its Phase-1 counterpart. The selection of interesting physics events at the Level-1 (L1) trigger and inefficiency of selection algorithms in high pileup conditions further require the Tracker to be included in this trigger stage, helping reduce the event rate from 40 MHz rate to 7.5 kHz. The physics goals also require an increase in Pixel Detector coverage to $|\eta| = 4.0$ which improves the $p_{\text{T}}^{miss}$ resolution and particle-flow event reconstruction by providing $p_\text{T}$ measurements and trajectories for charged particles entering the calorimeters. The $p_{\text{T}}^{miss}$ resolution is an essential performance parameter for many BSM physics searches including SUSY and extra dimension models where particles escape undetected from the detector space. The smaller pixel size will further improve b-tagging as well as hadronic reconstruction and track reconstruction efficiencies within boosted jets, which can be produced from new heavy objects decaying to Higgs, Z bosons, or top quarks -- all heavy probes that can be exploited for new physics searches. Improving the b-tagging capabilities directly affects our analysis due to the importance of properly identifying top quark decays.\\

\begin{figure}[H]
\begin{center}
\includegraphics[width=0.8\textwidth]{PixelDet.png} 
\caption{Phase-2 Pixel Detector Layout[ref].}
\label{PixelDet} 
\end{center}
\end{figure}

The Phase-2 Pixel Detector\cite{Phase2Tracker,Phase2Tracker2} baseline design comprises a barrel part with 4 layers of Tracker Barrel Pixel Detector (TBPX), 8 small double-discs per side of Tracker Forward Pixel Detector (TFPX) and 4 large double-discs per side of Tracker Endcap Pixel Detector (TEPX). This forward and end part is referred to as 8l4s (8 TFPX and 4 TEPX). In the TBPX the pixel modules are arranged in ``ladders''. In each layer, neighboring ladders are mounted staggered in radius, so that $r$-$\phi$ overlap between the ladders is achieved. The modules on a ladder do not overlap in $z$. A projective gap at $\eta = 0$ is avoided by mounting an odd number of modules along $z$, and by splitting the barrel mechanics in $z$ into slightly asymmetric halves. In TFPX and TEPX the modules are arranged in concentric rings. Each double-disc is physically made of two discs, which facilitates to mount modules onto four planes, with overlaps in $r$ as well as $r$-$\phi$. Each disc is split into two halves, and these D-shaped structures are referred to as ``dees''. The TEPX will provide the required luminosity measurement capability, by an appropriate implementation of the readout architecture. In total, the pixel detector will have an active surface of approximately 4.9 m$^2$. \autoref{PixelDet} shows the layout of the Phase-2 Pixel Detector.







\chapter{What is Machine Learning?\label{ML}}

Machine Learning (ML) can be defined as an application of Artificial Intelligence that permits the computer system to learn without being told explicitly. 
In ML a computer program is said to learn from experience E with respect to some class of tasks T and performance measure P, if its performance at tasks in T, as measured by P, improves with experience E \cite{Coursera}.
 ML has made tremendous strides in the past decades and has become very popular recently due to its multifaceted applications. It is being used on social media, marketing, and in the scientific community as well. 
Some examples of ML applications are: the algorithms used on application in smartphones to detect human faces, self-driving cars, computer games, stock prediction, and voice recognition. An interesting characteristic of ML algorithms is that the more data one inputs the better is the performance. The ML application has a very wide spectrum covering almost every aspect of human endeavor that involves a lot of data. 
Scientific analysis today generates enormous data and is a hence is a perfect used case to apply ML techniques. In this work we use enhanced ML techniques based on progress in the recent past.

In general, there are two main categories to classify machine learning problems: \textbf{Supervised Learning} (SL) and \textbf{Unsupervised Learning} (UL). SL is the most used ML approach and has proven to be very effective for a wide variety of problems. Examples of common SL problems are: spam filters, predicting housing prices, identifying a malignant or benign tumor, etc. These types of problems are characterized by providing a “right answer” as a reference. For example, spam filter algorithms identify emails that are spams by training on a dataset that has examples of such emails. In case of predicting house prices, the algorithm is trained on a dataset of houses involving features like the area of the house, number of rooms, and the selling price of the house.

UL algorithms are different in the sense that they do not have the “right answers” given to the machine. Instead, UL algorithms are used for finding patterns and make clusters from the given data. That is what also forms the basis of a search engine (e.g. Google news). Clicking on a link to a news article, one gets many different stories of different journals that have some correlation with the article searched. This happens because the ML algorithm is capable of learning features and shared patterns from a bunch of data without being given any specifics. Another interesting UL problem is the so-called “cocktail party” that involves distinguishing the voice of two people recording on two microphones located at different places. The ML algorithm is able to separate the sources of the voices in the recordings by learning the voice features that correspond to each person, showing the power of the UL algorithm.

In this study, I have focused on an SL approach and a variant of the UL approach, called the \textbf{Semi-Supervised Learning} approach (SSL). The SSL is named so because the data involves looking at images that are already known to be “Good” but one doesn’t necessarily know every possible situation that produces a “Bad” image. The purpose is to define a metric for a “good” image and subsequently decide if an image is “bad” in case it deviates too much from an acceptable value.



\section{Developing the Algorithm}

To develop an ML algorithm the following are taken into consideration, what is the task? and what is the method to approach the task? In our case, we are looking into images that have information about the activity that the channels in the HCAL are detecting. These images are called "occupancy maps" and they are a visual way of monitoring the health of the detector itself (see \autoref{Occupancymaps5x5}). There are two common problems that can be identified by viewing occupancy maps which are called "dead channels" and "hot towers". They are referred to as “\textbf{dead}” and “\textbf{hot}” respectively in the rest of this document. Dead channels mean that on a certain place in the occupancy map there is not any readout from the channels on the HCAL and hot channels mean that there are channels that are being triggered by noise or are damaged in a way that makes them readout too much activity.

\begin{figure}[h]
\begin{subfigure}{.3\textwidth}
	\centering
	\includegraphics[width=.8\textwidth]{Good_image.jpg}
	\caption{Good Image\label{Goodimage}}
\end{subfigure}
%
\begin{subfigure}{.3\linewidth}
	\centering
	\includegraphics[width=.89\linewidth]{Dead_image_5x5.jpg}
	\caption{Dead Image \label{Deadimage}}
\end{subfigure}
%
\begin{subfigure}{0.3\linewidth}
	\centering
	\includegraphics[width=.84\linewidth]{Hot_image_5x5.jpg}
	\caption{Hot Image \label{Hotimage}}
\end{subfigure}
\vspace{1cm}
\caption{Occupancy maps with 5x5 affected regions \label{Occupancymaps5x5}}
\end{figure}

The problem is the following, to create a model that can detect and classify what type of scenario is occurring on each occupancy map. For this, we want to go with a SL approach which means that we will give the model the images as the input and it will train on these images by learning to identify patterns or features in the image and try to do a “fit” from the images to their corresponding labels. After the training, the algorithm will be given a testing set for us to evaluate the model’s ability to correctly detect if there is a problem with the image and what type of problem is being detected. The output of the model will be the predicted class of the test image. The predictions are based on the labels and their corresponding images that were given to the model during training. This means that if the model was trained with 3 different types of images with their corresponding label the model will only work well for images that present similar patterns or characteristics to those presented in the training. For example, if we only train the model to distinguish between “good” and “hot” then when the model encounters images that aren’t either of these two, like an image labeled “dead”, then the model will not know what to do with this image and will give it an incorrect label.
After the SL model has been tested the next step is trying an SSL model. The term semi-supervised simply means that there isn’t a ground truth label that is being given to the model during training because either there isn’t necessarily a ground truth, or we don’t know what the ground truth is. What we do know, is what is considered as a “good” image and what this approach hopes to accomplish is to use the error in the reconstruction of the input image and use that information to discriminate between the “good” vs the “bad” images.

\section{Teaching the Algorithm}

The way an ML algorithm learns is by an iterative process called an optimization algorithm in which the predicted output value of the model is compared to the desired output (See \autoref{WandB}) and the weights and biases of the model are adjusted such that the predicted output is closer to the desired output.

\begin{figure}[th]
\centering
\includegraphics[width=.68\textwidth]{Weights_and_biases.png}
\caption{Weights and Biases \label{WandB}}
\end{figure}

“Optimization algorithms helps us to \textbf{minimize} (or \textbf{maximize}) an \textbf{Objective} function
\textit{(another name for \textbf{Error} function)} \textbf{E(x)} which is simply a mathematical function dependent on the Model’s internal \textbf{learnable parameters}
which are used in computing the target values(\textbf{Y}) from the set of \textit{predictors}(\textbf{X}) used in the model. 
For example - we call the \textbf{Weights(W)} and the \textbf{Bias(b)} values of the neural network as its internal learnable \textit{parameters} 
which are used in computing the output values and are learned and updated in the direction of optimal solution i.e. 
minimizing the \textbf{Loss} by the network’s training process and also play a major role in the \textit{\textbf{training}} process of the Neural Network Model.” \cite{optimizaiton_algorithms}.
The most basic and probably the most used optimizer is called Gradient Descent (GD). 
GD is based on the concept of using the gradient of a loss or cost function and moving the weights and biases of the ML model so that the predicted value is taking a step in the decreasing direction of this error function 
(See \autoref{fig:GD}). In general, the “terrain” of the loss function is not a smooth bowl-shaped surface like the one present in the image. The most general form of the surface is more similar to a rocky mountain (See \autoref{fig:LF}), which presents a problem when using simple optimizers like GD.


\begin{figure}
\begin{center}

\includegraphics[width=.8\textwidth]{GD.png}
\caption{Gradient Descent algorithm\label{fig:GD}}
\end{center}

\end{figure}



\begin{figure}
\begin{center}

\begin{subfigure}{.9\linewidth}
\includegraphics[width=\textwidth]{Loss_function1.png}
\end{subfigure}

\begin{subfigure}{.9\linewidth}
\includegraphics[width=\textwidth]{Loss_function2.png}
\end{subfigure}

\caption{Loss Function surface\label{fig:LF}}
\end{center}

\end{figure}




\chapter{Results \label{ch:results}}
Here first the limitations of Scikit-learn predefined ML models - Logistic Regression(LR) and Multi-Layer-Perceptron(MLP), are described. The Logistic Regression Model seems to work almost perfectly with all 3 classes when the bad region size is $5\times5$ (as in \autoref{Occupancymaps5x5}) with either the same or randomized location. When the bad region size is 1x1 like in \autoref{Ocuppancymaps1x1} the LR Model performs poorly with an accuracy of approximately 20\%.
 The MLP does not seem to work in any of the used cases that are studied as it always performs poorly with an accuracy of $\approx 40\%$.

\begin{figure}[h]
\centering
\begin{subfigure}[t]{.316\textwidth}
\includegraphics[width=\textwidth]{Good_image_1x1.png}
\caption{}
\end{subfigure}
\begin{subfigure}[t]{.305\textwidth}
\includegraphics[width=\textwidth]{Dead_image_1x1.png}
\caption{}
\end{subfigure}
\begin{subfigure}[t]{.3\textwidth}
\includegraphics[width=\textwidth]{Hot_image_1x1.png}
\caption{}
\end{subfigure}
\vspace{5mm}
\caption{Occupancy Maps with 1x1 bad regions. A) Good image B) Dead image C) Hot image\label{Ocuppancymaps1x1}}
\end{figure}

Also, the use of Scikit-learn’s library is limited in comparison with the Keras module since one cannot customize the structure of the ML model with detail. Moreover, Keras is an ML library designed for developing deep neural networks. Hence it was decided to use Keras primarily for the creation of the model.
 With the Keras library, numerous models were designed with both, SL method and SSL learning method. Using SL method, we are interested in detecting anomalies and classifying what type of anomaly is seen. With SSL method, we are interested in looking at the error of the reconstruction of an image to give an idea that the image given can be considered good or that it might have some unseen anomalies

\section{SL Models for known anomalies in the HCAL data for DQM}
We considered three SL Models for classification of known anomalies in the HCAL data for DQM. 
These models are based on Convolutional Neural Networks and differ in the number of layers utilized, their ordering and number of units in each layer. The Models and the corresponding results are described below.


\subsection{Two Convolutional Layers for binary classification}
Several variations of the two Convolutional Layers Model were tested and optimized on the DQM data. This led to an optimal value of 8 units/neurons in the Convolutional layers. 
The detail of selecting the number of units per layer is of great importance to find a balance between efficiency and complexity of a model. More complex models (more layers and connections) are “heavy” to train in terms of computational cost, provide better results and are prone to “overfitting” to the training data.
 Simpler models (fewer layers and connections) are quicker to train, efficient and computationally economic. However, simpler models are more likely to “underfit” to the data. The \autoref{fig:2convlayermodel} below shows a code snippet with this model.
\autoref{fig:2convlayermodelfixedresults} below shows the learning curve for this model trained with Good and Hot images for fixed $5\times 5$ location and the corresponding Confusion Matrix.

\autoref{fig:2convlayermodelrandomresults} shows the learning curve for this model trained with Good and Hot images for fixed $5\times5$  location and the corresponding Confusion Matrix.

\begin{figure}
\begin{center}
    \includegraphics[width=.9\textwidth]{2_conv_layers_model.png}
\end{center}
\caption{Two Convolutional Layers Model\label{fig:2convlayermodel}}
\end{figure}


\begin{figure}
\centering
	\begin{subfigure}{.45\textwidth}
 	\includegraphics[width=\textwidth]{CM_2x2_with_5x5_good_hot_fixed.png}
	\end{subfigure}
	\begin{subfigure}{.45\textwidth}
	\includegraphics[width=\textwidth]{Learning_curve_5x5_good_hot_fixed.png}
	\end{subfigure}
	\caption{Confusion Matrix results and Learning curve
	 for $5\times 5$ damaged area with on the same location for all trials\label{fig:2convlayermodelfixedresults}}
 \end{figure}
 
 \begin{figure}
 	\begin{subfigure}{.45\textwidth}
 		\includegraphics[width=\textwidth]{2x2CMwith_5x5_good_hot_random.png}
 	\end{subfigure}
 	\begin{subfigure}{.45\textwidth}
 	\includegraphics[width=\textwidth]{Learning_curve_5x5_good_hot_random.png}
 	\end{subfigure}
 \caption{Confusion Matrix results and Learning curve
	 for $5\times5$ damaged area with on the random location for all trials\label{fig:2convlayermodelrandomresults}}
 \end{figure}
 

\begin{figure}
	\begin{subfigure}{.5\textwidth}
 		\includegraphics[width=\textwidth]{3x3CMwith_5x5_good_hot_dead_random.png}
 	\end{subfigure}
 	\begin{subfigure}{.45\textwidth}
 	\includegraphics[width=\textwidth]{Learning_curve_5x5_good_hot_dead_random.png}
 	\end{subfigure}
 \caption{Confusion Matrix results and Learning curve
	 for $5\times5$ damaged area with an extra class to identify with random location for all trials\label{fig:2convlayermodelGHDrandomresults}}
 \end{figure}
 
\autoref{fig:2convlayermodelGHDrandomresults} shows the learning curve for this model trained with Good, Hot and Dead images for random $5\times5$ location and the corresponding Confusion Matrix

\autoref{fig:2convlayermodelGHD1x1randomresults} shows the learning curve for this model trained with Good, Hot and Dead images 
for random 1x1 location and the corresponding Confusion Matrix. 
The corresponding learning curves and confusion matrix for a fixed location for 3-class 
(Good, Hot, Dead) configuration give the same behavior as 2-labels (Good, Hot) images



\begin{figure}
\begin{subfigure}{.5\textwidth}
\includegraphics[width=\textwidth]{3x3CMwith_1x1_good_hot_dead_random.png}
\end{subfigure}
\begin{subfigure}{.45\textwidth}
\includegraphics[width=\textwidth]{Learning_curve_1x1_good_hot_dead_random.png}
\end{subfigure}
\caption{Confusion Matrix results and Learning curve for $1\times1$ damaged area with random location for all trials
 \label{fig:2convlayermodelGHD1x1randomresults}}
\end{figure}
	
	
In a more realistic scenario, the problems with HCAL DQM would be more granular i.e. $1\times1$ type. 
When this model is tested for problematic channels in $1\times1$ configuration the learning curves for the training (blue) and validation (orange) sets depart after few epochs as shown in Figure 12 (right part).
From the left part of the figure, dividing the sum of numbers along the diagonal (377+398+36) by the sum of all the numbers in the matrix gives ~1/3. This demonstrates that the model is “overfitting” to the training set and misclassifies images $\approx 33\% $ of times. Hence, we consider adding a Convolutional layer to gain more prediction accuracy as shown in the next section.	
	
	
	
	
In this section the results for the final estimation of the of the Z$\rightarrow\nu\bar{\nu}$ are presented. The current study includes preliminary results using only data obtained at the CMS detector during 2016. The results for this study are intended to confirm the assumption that the additional $\gamma+$jets control region introduced in this analysis reduce the overall uncertainties obtained in the 2016 analyses (described in \autoref{AnalysisChap}). Furthermore, this study is intended as a benchmark for future analyses of the SUSY stop group based in Fermilab and will be the method used for the 2017 CMS data.

\subsection{Systematics}\label{systematics}

Two categories of uncertainties for the Z$\rightarrow\nu\bar{\nu}$ prediction are considered: uncertainties that are associated to the use of MC simulation and the uncertainties specifically associated to the background prediction method. Several sources are acknowledged in the first category mentioned such as PDF and renormalization/factorization scale choices, jet and $p_\text{T}^{miss}$ energy scale uncertainties b-tag scale factor uncertainties, and trigger efficiency uncertainties. Given that the simulation sample is normalized to data in the tight control region, uncertainties associated with the luminosity and cross-section are excluded. In addition, the overall Z$\rightarrow\nu\bar{\nu}$ statistical uncertainty from MC simulation is also taken into account.\\

\begin{figure}[H]
\begin{center}
\includegraphics[width=0.8\textwidth]{UncZnunu}
\end{center}
\vspace{-1em}
\caption{Systematic uncertainty in the final prediction, as a function of the search bin, associated to the MC statistics.}
\label{UncZnunu}
\end{figure}

The statistical uncertainty associated with each bin in the MC is propagated as a systematic uncertainty. The relative uncertainty per bin can be see in \autoref{UncZnunu}. It shows that the uncertainties for the MC vary from as low as 1\% up to 81\% and even 100\% in some regions. Since the final estimation is scaled using the global normalization factor from the tight $\mu\mu$ control region ($R_{norm}$), the total uncertainty, due to limited amounts of events in data, is propagated in the final prediction. This is also true for the $S_\gamma$($N_j$) scale factor, in which the residual differences in search variables other than $N_j$ are evaluated in the loose photon control region. Both the uncertainty arising from the $N_j$ re-weighting as well as the residual differences are evaluated together. The uncertainty from $R_{norm}$ is propagated as a flat value of 7.9\% uncertainty per each search bin.

\subsection{Z$\rightarrow\nu\bar{\nu}$ Estimation for the Search Bins}

The final estimation for the Z$\rightarrow\nu\bar{\nu}$ background calculated for all 84 search bins is shown in \autoref{results}. The statistical uncertainty in bins that have zero events is treated as the average weight (the sum of the weights squared over the weight) times the poisson error on 0 which is 1.8. This average weight is calculated on the basis of a relaxed cut in which $N_b \geq 2$ is required. For comparison, a cut in which $N_t > 2$ where two tops are fake for the Z$\rightarrow\nu\bar{\nu}$ is used.

\begin{figure}[H]
\begin{center}
\includegraphics[width=0.8\textwidth]{Results.png}
\end{center}
\vspace{-1em}
\caption{Z$\rightarrow\nu\bar{\nu}$ background prediction for all search bins, including the breakdown of the various uncertainties.}
\label{results}
\end{figure}



%\chapter{Conclusion}\label{conclusion}
%\input{chapters/conclusion}


\appendix
%\chapter{Appendix Title}
%%\input{chapters/appendix}
\nocite{*}
\bibliography{references}
\bibliographystyle{ieeetr}

\end{document}
